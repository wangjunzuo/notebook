\documentclass[a4paper]{article}
\usepackage{ctex}
\usepackage[margin=1in,top=1cm]{geometry}
\usepackage{listings}
\lstset{tabsize=3}

\title{yarn}
\author{W.J.Z}
\date{2019.4.18}

\begin{document}
	\maketitle
	\section{yarn运行机制}
	YARN是Hadoop的集群资源管理系统,通过两类长期运行的守护进程提供自己的核心服务:resource manager和node manager。node manager运行在集群所有节点,负责启动和监控容器,容器是用于执行特定应用程序的进程。
	\subsection{资源请求}
	YARN有一个灵活的资源请求模型,当请求多个容器时,可以指定每个容器需要的计算机资源数量(内存和CPU数量),还可以指定对容器的本地限制要求。本地限制可用于申请指定节点或机架、或集群中任何位置的容器。YARN应用可以在运行中的任意时刻提出资源申请。
	
	通常情况下,启动一个容器处理HDFS数据块时,应用将会向这样的节点提出申请:储存该数据块三个副本的节点,或是储存这些副本的机架中的一个节点。如果都申请失败,则申请集群中任意一个节点。
	
	\subsection{应用生命周期}
	YARN应用的生命周期差异性很大,可分为下面三个模型:(1) 一个用户作业对应一个应用,MapReduce采取这种方式。(2) 作业的每个工作流和每个用户对话对应一个应用,容器可以在作业间重用且可能缓存中间数据,Spark采用该模式。(3)多个用户共享一个长期运行的应用,这种应用通常作为一种协调者的角色在运行。
	
	\section{Yarn调度}
	理想情况下,YARN应用发出的资源请求应立刻给与满足,但现实情况下资源是有限的,在一个繁忙的集群上,一个应用经常需要等待才能得到所需的资源,YARN提供多种调度器和配置策略龚我们选择的原因。
	
	\subsection{调度选项}
	YARN有三种调度器可用:FIFO调度器、容量调度器、公平调度器。
	\begin{enumerate}
		\item FIFO Scheduler:将应用放在一个队列中,按照提交的顺序运行应用。该调度器简单易懂,不需要任何配置,但不适用于共享集群。
		
		\item Capacity Scheduler:预留一部分资源并设立一个独立的专门队列负责小作业,由于牺牲了整个集群的利用率,大作业执行的时间要更长。
		
		\item Fair Scheduler:调度器会在所有运行的作业之间动态平衡资源。
	\end{enumerate}
	
	
	\subsection{容量调度器配置}
	容量调度器允许多个组织共享一个Hadoop集群,每个组织可以分配到全部集群资源的一部分。每个组织被分配一个专门的队列,每个队列被配置为可以使用一定的集群资源。队列可以进一步按层次划分,这样每个组织内的不同用户能共享该组织队列所分配的资源。 在一个队列内,使用FIFO调度策略对应用进行调度。
	
	容量调度器配置文件:capacity-cheduler.xml,对特定队列进行配置时:yarn-scheduler.capacity.<queue-path>.<sub-property>进行设置,<queue-path>表示队列的层次路径(用圆点隔开)。
	root队列下prod和dev两个队列,dev下有eng和science两个队列,配置条件如下:
	\begin{lstlisting}
	<?xml version="1.0">
	<configuration>
		<property>
			<name>yarn.scheduler.capacity.root.queues</name>
			<value>prod,dev</value>
		</property>
		<property>
			<name>yarn.scheduler.capacity.root.dev.queues</name>
			<value>eng,science</value>
		</property>
		<property>
			<name>yarn.shceduler.capacity.root.prod.capacity</name>
			<value>40</value>
		</property>
		<property>
			<name>yarn.schedul;er.capacity.root.dev.capacity</name>
			<value>60</value>
		</property>
		<property>
			<name>yarn.shceduler.capacity.root.dev.maxinum-capacity</name>
			<value>75</value>
		</property>
		<property>
			<name>yarn.scheduler.capoacity.root.dev.eng.capacity</name>
			<value>50</value>
		</property>
		<property>
			<name>yarn.shceduler/capacity.root.dev.science.capacity</name>
			<value>50</value>
		</propety>
	</configuration>
	\end{lstlisting}
	
	将应用放置那个队列取决于应用本身:在MapReduce中,可以通过设置属性Mapreduce.job.queuename来指定要使用的队列。如果队列不存在,在提交时会发送错误。如果未指定队列,name应用将放在一个名为default的默认队列中。
	
	\subsection{公平调度器配置}
	\subsubsection{启用公平调度器}
	hadoop默认使用容量调度器,启用公平调度器需要将yarn-site.xml文件中的yarn-resourcemanager.sche\\duler.class设置为公平调度器的完全限定名:org.aoachehadoop.yarn,server.resourcemanager.shceduler.fair.FairScheduler.
	
	\subsubsection{队列配置}
	公平调度器通过对fair-scheduler.xml文件进行配置:
	\begin{lstlisting}
	<?xml version="1.0">
	<allocations>
	<defaultQueueSchedulingPolicy>fair</defaultQueueSchedulingPolicy>
	<queue name="prod">
		<weight>40</weight>
		<schedulingPolicy>fifo</schedulingPolicy>
	</queue>
	
	<queue name="dev">
		<weight>60</weight>
		<queue name="eng" />
		<queue name="science" />
	</queue>
	
	<queuePlacementPolicy>
		<rule name="specified" create="false" />
		<rule name="primaryGroup" create="false" />
		<rule name="default" queue="dev.eng" />
	</queruePlacementPolicy>
	\end{lstlisting}
	每个队列可以有不同的调度策略,队列的默认调度策略可以通过顶层元素defaultQueueSchedulingPolicy进行设置,如果省略默认使用公平调度。队列的调度策略可以被该队列的schedulingPolicy元素指定的策略覆盖。
	
	\subsubsection{队列配置}
	queuePlacementPolicy元素包含了一个规则列表,每条规则会被依次尝试指导匹配成功。如果没用明确定义队列,则按照用户名为队列名进行创建。
	\begin{lstlisting}
		<queuePlacementPolicy>
			<rule name="specified" />
			<rule name="user" />
		</queuePlacemenPolicy>
	\end{lstlisting}
	
	\subsubsection{延迟调度}
	所有的YARN调度器都已本地请求为重,为了增加集群的工作效率,可以等待一小段时间进行分配容器。
	
	YARN中的每个节点管理器周期性的向资源管理器发送心跳请求,心跳中携带了节点管理器中正运行的容器,新容器可用资源等信息。对于需要运行一个容器的应用来说,每个心跳就是一个潜在的调度机会。
	
	使用延迟调度器时,调度器不会简单使用它收到的用第一个调度机会,而是等待设定的最大数目的调度机会发生,然后才放松本地性限制并接受下一个调度机会。
	
	对于容量调度器:通过设置yarn.scheduler.capacity.node-locality-delay来配置延迟调度。配置为正整数,表示调度器在放松节点限制、改为匹配同一机架上的其他节点之前,准备错过的调度机会的数量。
	
	对于公平调度器:yarn.scheduler.fair-locality.threshold.node设置为0.5,表示调度器在接受统一机架中的其他节点之间,将一直等待直到集群中的一半节点都已经给过调度任务。yarn.scheduler.fair.\\locality.threshold.track表示在接受另一个机架替代所申请的机架之前需要等待的时长阈值。
	
	\subsubsection{主导资源公平性}
	对于多种资源调度时,如何衡量调度公平性?Yarn采用DRF:观察每个用户的主导资源,并将其作为对集群资源使用的一个度量(对比使用占比最大的资源)。默认情况下,YARN只考虑内存,不考虑CPU。启用DRF对容量调度器:将capacity-scheduler.xml文件中的org.apache.hadoop.yarn.util.\\resource.DominantResourceCalculator设置为yarn.scheduler.capacity.resource-calculator即可。对公平调度器设置分配文件中的顶层元素;defaultQueueSchedulingPolicy为drf即可。
\end{document}