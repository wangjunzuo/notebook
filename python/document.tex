\documentclass[twocolumn]{article}
\usepackage{ctex}
\usepackage{amsmath}
\usepackage{ graphics}
\usepackage{listings}
\usepackage{multicol}
\graphicspath{{picture/}}
\title{python知识点总结}
\author{W.J.Z}
\date{}
\lstset{xleftmargin=-3em,language=python}
\begin{document}
	\maketitle

	
	\section{基础知识}
	\subsection{python必备知识}
	\begin{enumerate}
		\item 12个数学操作符: +、 -、 *、 /、 \%、 \///、 **、 +=、 -=、 *=、 /=、 \%=。 
		
		\item 2个基本函数:print()、input()。
		
		\item 12个逻辑符号:True、False、==、!=、<、>、<=、>=、and、or、not。
	\end{enumerate}

	\subsection{控制语句}
	if 控制语句:

	\begin{lstlisting}
	age = 10
	if age < 10:
		print('low')
	elif age > 10:
		print('high')
	else:
		print('right')
	\end{lstlisting}
	
	while控制语句:
	\begin{lstlisting}
	age = 1
	while age < 5:
		age = age +1
	print(age)
	\end{lstlisting}
	
	for 控制语句
	\begin{lstlisting}
	for i in range(1,10,2):
		print(i)
	\end{lstlisting}
	
	\subsection{函数定义}
	\begin{lstlisting}
	def function(num)
	while num < 5:
		print(num)
		num = num + 1
	return num
	new = function(1)
	print(new)
	\end{lstlisting}
	
	\subsection{数据结构}
	\subsubsection{列表}
	列表常用方法:
	\begin{lstlisting}
	parm = [1,2,3,4]
	parm[3]  #查看指定下标的数据  
	parm[-1]   #倒着数列表   
	parm[1:3]  #对数据进行切片
	len(parm)#len()函数求列表的长度
	parm * 2      #重复2遍列表
	parm + [5,6]  #列表连接  
	del parm[3]#删除指定坐标的数据
	parm.index(2) #获取数据2所在的下标
	parm.append(4)#末尾添加数据4
	parm.insert(2,2)#在下标2插入数据2
	parm.remove(2)#删除第一个2
	parm.sort(reverse=True)#逆序排列列表
	\end{lstlisting}
	\subsection{元组}
	元组数据内容不能被修改、添加、删除,其他函数方法与列表相同。将数据转换为列表和元组使用list()和tuple()函数。
	\begin{lstlisting}
	list((1,2,3,4))
	tuple([1,2,3,4])
	\end{lstlisting}
	\subsection{字典}
	\begin{lstlisting}
	parm = {'name':'xiaoli','age':'12'}
	parm.keys()  #键
	parm.values() #值
	parm.items()  #所有数据
	#返回name对应的值.没有返回0
	parm.get('name',0)
	#设置默认sex
	parm.setdefault('sex','female')
	\end{lstlisting}
	\subsection{字符串}
	
	\begin{tabular}{|c|p{0.3\textwidth}|}
		\hline
		方法 & 说明 \\
		\hline
		upper() & 所有字母变为大写 \\
		\hline
		lower() & 所有字母变为小写 \\
		\hline
		isupper() & 字符串是否为大写 \\
		\hline
		islower() & 字符串是否为小写 \\
		\hline
	\end{tabular}
	\\
	\\
	upper()和lower()函数会创建新的副本,而不是直接修改的原有数据,如果需要更改使用pparm = parm.lower()方式。
	\\
	
	\begin{tabular}{|c|p{0.3\textwidth}|}
		\hline
		方法 & 说明 \\
		\hline
		isalpha() & 字符串只包含字母,非空 \\
		\hline
		isalnum() & 只包含字母和数字,非空 \\
		\hline
	    isdecimal() & 只包含数字字符,非空 \\
		\hline
		isspace() & 只包含空格、制表符、换行,非空\\
		\hline
		istitle() & 大写字母开头,后面都是小写字母 \\
		\hline
		startswith() & 字符串开头的单词判断\\
		\hline
		endswith()  & 字符串结尾单词判断 \\
		\hline
		join() & 连接字符串列表\\
		\hline
		split() & 切割字符串 \\
		\hline
		rjust() &参数一指定从右对齐长度,参数二指定填充字符 \\
		\hline
		ljust() &参数一指定从左对齐长度,参数二指定填充字符 \\
		\hline
		cjust() &参数一指定从左右对齐长度,参数二指定填充字符 \\
		\hline
		strip() & 删除字符左右的空白字符 \\
		\hline
		rstrip() & 删除字符右的空白字符 \\
		\hline
		lstrip() & 删除字符左的空白字符 \\
		\hline
	\end{tabular}
	\section{正则表达式}
	\subsection{正则表达式速记技巧}
	$. \quad [\quad] \quad \wedge \quad \$ $是所有语言都支持的正则表达式,正则难理解因为其有一个等价概念,将等价恢复于原始写法就简单的多。

	\noindent 1、等价

	?,*,+,$\setminus$d,$\setminus$w都是等价字符。	

	?等价于匹配长度\{0,1\} 
	
	*等价于匹配长度\{0,\}
	
	+等价于匹配长度\{1,\}
	
	$\setminus$d等价于$[0-9]$
	
	$\setminus$D等价于$[^\land0-9]$
	
	$\setminus$w等价于$[A-Za-z0-9]$
	
	$\setminus$W等价于$[^\land A-Za-z0-9]$
	\\
	
	\noindent 2、常用运算符与表达式
	
	$^\land$开始
	
	()域段
	
	[\quad]包含
	
	[$^\land$\quad]不包含
	
	\{m,n\}匹配长度
	
	$.$ 任何单个字符
	
	$|$ 或
	
	$\setminus$转义
	
	$\$$结尾
	
	[A-Z] 26个大写字母
	
	[a-z] 26个小写字母
	
	[0-9] 0到9数字
	
	[A-Za-z0-9] 
	
	,分割
	
	[0,3]包含0或3数字
	
	\noindent 3、语法与释义
	
	基础语法$ "^\land([]\{\})([]\{\})\$"$
	
	技巧:?,*,+,$\setminus$d,$\setminus$w 这些都是简写的,完全可以用[]和\{\}代替
	
	\subsection{python正则表达式}
	\begin{enumerate}
		\item 用import re导入正则表达式模块
		\item 用re.complie()函数创建一个Regex对象
		\item 向Regex对象的search()方法传入向查找的字符串
		\item 调用Match对象的group()方法返回实际匹配的字符串
	\end{enumerate}
%	\lstset{xleftmargin=-3.5em}
	\begin{lstlisting}
	括号分组
	regex=re.compile(r'(\d{3})-(\d{3}-\d{4})')
	mo=regex.search('my num 415-555-4344')
	mo.group() #415-555-4344
	mo.group(1)#415
	mo.group(2)#555-4344
	
	匹配多个分组,和c语言的或作用一样
	regex = re.compile(r'b |a')
	mo = regex.search('b')
	mo.group() #b
	mo = regex.search('a')
	mo.group() #a
	
	贪心与非贪心
	regex = re.compile(r'(ha){3,5}')
	mo=regex.search('hahahahaha')
	mo.group() #hahahahaha
	regex = re.compile(r'(ha){3,5}?')
	mo = regex.search('hahahahaha')
	mo.group() #hahaha
	
	findall()方法
	search()方法返回一个Match对象,包含
	被查找字符串中的第一次匹配的文本,而
	findall()方法将返回一组字符串。
	
	sub()替换字符串
	sub(arg1,arg2);参数一为替换字符串
	参数二为匹配内容
	
	\end{lstlisting}
	
	\section{读取文件}
	\subsection{路径}
	\begin{lstlisting}{language=python}
	import os
	#根据系统获取正确的路径
	os.path.join('usr','bin')
	#获取当前工作路径
	os.path.getcwd()
	#改变当前工作路径
	os.path.chdir('c:\\')
	#创建新文件夹
	os.makedirs('c:\\newdir')
	#获取绝对路径
	os.path.abspath('.')
	#是否为绝对路径
	os.path.isabs(path)
	#返回从start到path的相对路径
	os.path.relpath(path,start)
	#获取路径名
	os.path.dirname(path)
	#获取文件名
	os.path.basename(path)
	#获取文件大小
	os.path.getsize(path)
	#获取文件夹下文件名列表
	os.listdir(path)
	#检查路径是否存在
	os.path.exists(path)
	#检查文件是否存在
	os.path.isfile(path)
	#检查是否为文件夹
	os.path.isdir(path)
	\end{lstlisting}
	\subsection{读取文件}
	\begin{lstlisting}
	#打开文件,r读模式 w写模式 a追加模式
	file = open(path,'r')
	#读取文件
	content = file.read();
	content1 = file.readlines();
	#写文件
	file  = open(path,'w')
	file.write(text+'\n')
	file.close()
	\end{lstlisting}
	
	\section{组织文件}
	\subsection{文件和文件夹操作}
	\begin{lstlisting}
	import shutile,os
	#将spam.txt文件复制到dev文件夹下
	#并重命名pam.txt
	shutil.copy('c:\\spam.txt','c:\\dev\\pam.txt')
	#复制整个文件夹
	shutil.copytree('c:\\bacon','c:\\backup')
	#将pam文件移动到eggs文件夹重命名spam.txt
	shutil.move('c:\\pam.txt','c:\\eggs\\spam.txt')
	#删除文件夹和文件
	shutil.rmtree(path)
	#使用send2trash安全删除
	import send2trash
	send2trash.send2trash(file)
	#遍历目录树
	os.walk();
	#读取zip文件
	improt zipfile,os
	zip=zipfole.ZipFile(file)
	#获取压缩包李文件名列表
	zip.namelist()
	#加压zip文件
	zip.extractall()
	#指定解压文件
	zip.extract(file)
	#创建zip文件
	zip = zipfile.ZipFile(file,'w')
	zip.write('filename',compress='')
	zip.close()
	\end{lstlisting}
	
	\section{调试}
	\subsection{抛出异常}
	\begin{lstlisting}
	def demo(num):
	if num != 1:
 		raise Exception("num")
		
	for i in range(0,3,1):
	try:
		demo(i)
	except Exception as err:
		print(str(err))
	\end{lstlisting}
	
	\subsection{日志模块}
	
	\begin{lstlisting}{language=c}
	import logging
	logging.basicConfig(filename='log.txt',
	level=logging,DEBUG,format=
	'%(asctime)s-%(levelname)s-%(message)s')
	logging.disable(logging.DEBUG)
	\end{lstlisting}
	
	\noindent python的日志级别
	
	\begin{tabular}{|c|c|c|}
		\hline
		级别 &日志函数 &描述 \\
		\hline
		DEBUG &loggging.debug() &最低级别 \\
		\hline
		INFO & logging.info() &记录程序中一般时间\\
		\hline
		WARNING &logging.warning() & 表示可能的问题\\
		\hline
		ERROR & logging.error() & 记录错误 \\
		\hline
		CRITICAL & logging.critical() & 表示致命错误\\
		\hline 
	\end{tabular}
	\section{web爬虫}
	\subsection{webbrowser}
	\begin{lstlisting}
	import requests
	res = requests.get('http://www.baidu.com/1/txt')
	#检查下载是否成功
	res.raise_for_status()
	#将下载内容写入文件
	file = open('2.txt','wb')
	for duff in res.iter_content(10000)
		playfile.write(chunk)
	file.close()
	import requests,bs4
	res = requests.get('www://d.com')
	res.raise_for_status()
	elem = bs4.BautifulSoup(res.text)
	#使用select()方法寻找元素
	elems = elem.select('span')
	#获取元素的属性值
	elems.get('id')
	\end{lstlisting}
	\onecolumn
	\subsection{selemium}{}
	selenium的WebDriver方法,用于寻找元素。
	\begin{lstlisting}
		#使用CSS类的name元素
		browser.find_element_by_class_name(name)
		browser.find_elements_by_class_name(name)
		#匹配CSS selector元素
		browser.find_element_by_css_selector(selector)
		browser.find_elements_by_css_selector(selector)
		#匹配id属性值得元素
		browser.find_element_by_id(id)
		browser.find_elements_by_id(id)
		#完全匹配提供的text的<a>元素
		browser.find_element_by_link_text(text)
		browser.find_element_by_link_text(text)
		#包含提供的text的<a>元素
		browser.find_element_by_partial_link_text(text)
		browser.find_elements_by_partial_link_text(text)
		#匹配name属性值得元素
		browser.find_element_by_name(name)
		browser.find_elements_by_name(name)
		#匹配标签name的元素
		browser.find_element_by_tag_name(name)
		browser.find_elements_by_tag_name(name)
	\end{lstlisting}
	\begin{lstlisting}
		from selenium import webdriver
		browser = webdriver.Firefox()
		browser.get('www.baidu.com')
		elem = browser.find_element_by_link_text('a')
		elem.click()
	\end{lstlisting}
	\twocolumn
	\section{处理execl文件}
	\begin{lstlisting}
	import openpyxl
	wb=openpyl.load_workbook('ee.xlxs')
	wb.get_sheet_names()
	sheet=wb.get_sheet_by_name('sheet1')
	sheet=wb.get_active_sheet()
	sheet['A1']
	sheet.cell(row=1,column=2)
	sheet['A1':'C3']
	sheet.columns[1]
	sheet.rows[1]
	wb.creat_sheet(index=2,title='a')
	#sheet为worksheet对象,不是字符串
	wb.remove_sheet(sheet)
	\end{lstlisting}
	\section{处理CSV和JSON文件}
	\noindent 处理csv文件:
	\begin{lstlisting}
	import csv
	file = open('file.csv')
	reader = csv.reader(file)
	data = list(reader)
	newfile = open('out.csv','w',
	newline='')
	output = csv.writer(newfile)
	output.writerow(data[1])
	output.close()
	\end{lstlisting}
	处理JSON文件:
	\begin{lstlisting}
	improt json
	data= json.loads(file.text)
	\end{lstlisting}
	\section{时间函数}
	\begin{enumerate}
		\item datetime对象包含一些整型值,保存在year、month、day、hour、minute、second等属性中。
		\item timedelta对象表示一段时间
		\item time.time() 返回当前时刻Unix纪元时间戳
		\item time.sleep(seconds) 让程序暂停seconds参数指定的秒数
		\item datetime.datetime(year,month,day,hour,minute,second)返回指定时刻的datetime对象
		\item datetime.datetime.now() 函数返回当前时刻的datetime对象
	\end{enumerate}
	\begin{lstlisting}
	import threading
	#APP为自定义函数,args为传参
	threading.Tread(target=APP,args=[])
	\end{lstlisting}
	\section{发送短信}
	\begin{lstlisting}
	from twilio.rest 
	improt TwilioRestClient
	accountSID='xxx'
	authToken='xxxx'
	twilioCli=TwilioRestClient(accountSID,
	authToken)
	myTwilioNumber='xxx'
	myCellPhone='xxx'
	message=twilioCli.messages.create(body=''
	,from_=myTwilioNumber,to=myCellPhone) 
	\end{lstlisting}
	\section{用gui自动化控制键盘和鼠标}
	\subsection{控制鼠标}
	\begin{lstlisting}
	import pyautogui
	#执行动作之前都会等待一秒
	pyautogui.PAUSE=1
	#获取屏幕的大小
	pyautogui.size()
	#移动鼠标
	pyautogui.moveTo(x,y,duration=0.25)
	#获取当前坐标信息
	pyautogui.position()
	#点击鼠标
	pyautogui.click(x,y)
	#拖动鼠标
	pyautogui.dragTo(x,y,duration=1)
	#从当前坐标拖动鼠标
	pyautogui.dragRel(x,y,duration=1)
	#滚动鼠标,200个单位
	pyautogui.scroll(200)
	#获取屏幕快照
	im=pyautogui.screenshot()
	#匹配坐标
	pyautogui.pixelMatchesColor(50,200,(r,n,g))
	im.getpixel((200,200))
	#图像识别,obj为目标图像
	pyautogui.locateOnScreen('obj.png')
	#查找所有匹配的目标
	list(pyautogui.locateAllOnScreen('obj.png'))
	#获取目标的坐标
	pyautogui.center((1,2,2,3))
	#点击
	pyautogui.click((1,2))
	\end{lstlisting}
	\subsection{控制键盘}
	\begin{lstlisting}
	#发送字符串
	pyautogui.typewrite('hello')
	#按键
	pyautogui.press('1')
	#热键组合
	pyautogui.hotkey('ctrl','c')
	\end{lstlisting}
\end{document}