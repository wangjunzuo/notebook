\documentclass{article}
\usepackage{ctex}
\usepackage{listings}
\lstset{language=C++,tabsize=2}
\usepackage[ruled]{algorithm2e}
\usepackage[top=2.5cm]{geometry}
\usepackage{amsmath}
\title{动态规划\&\&网易合唱团笔试题}
\author{W.J.Z}
\date{}

\usepackage{fancyhdr}
\pagestyle{fancy}
\lhead{}
\chead{}
% bfseries
\lhead{\bfseries W.J.Z copyright reserved }
\rhead{\bfseries https://blog.csdn.net/Leader\_wang}

\begin{document}
	\maketitle
	\section{问题描述}
	
	有 n 个学生站成一排,每个学生有一个能力值,牛牛想从这 n 个学生中按照顺序选取 k 名学生,要求相邻两个学生的位置编号的差不超过 d,使得这 k 个学生的能力值的乘积最大,你能返回最大的乘积吗? 
	
	
	每个输入包含 1 个测试用例。每个测试数据的第一行包含一个整数 n (1 <= n <= 50),表示学生的个数,接下来的一行,包含 n 个整数,按顺序表示每个学生的能力值 ai(-50 <= ai <= 50)。接下来的一行包含两个整数,k 和 d (1 <= k <= 10, 1 <= d <= 50)。
	\section{动态规划}
	\subsection{子问题分解}
	\textbf{分解方式:}从n个学生中选取k名学生>>从n个学生中选取1名学生;从n个学生中选取2名学生;\ldots;从n个学生中选取k名学生。

	\subsection{状态矩阵}
	两种分解方式的状态矩阵相同,状态总数为$k*n$个,令pMax(k,i)代表该状态下k个能力值乘机最大的结果,k代表从n个学生中选取k名学生且最后一名学生的标号为i,num(i)代表第i个学生的能力值.Pmax矩阵与子问题分解对应起来很好理解,至于为什么i为选取学生的最后一名的标号将在状态转移方程中讲解。
	\begin{table}[htbp]
		\centering
		\caption{pMax矩阵}
		\begin{tabular}{|c|c|c|c|c|}
			\hline
			0& 1&2 & \ldots & n \\
			\hline
			1&pMax(1,1)&pMax(1,2)&\ldots&pMax(1,n)\\
			\hline
			2&pMax(2,1)&pMax(2,2)&\ldots&pMax(2,n)\\
			\hline
			\ldots&\ldots&\ldots&\ldots&\ldots\\
			\hline
			k&pMax(k,1) & pMax(k,2)&\ldots&pMax(k,n) \\
			\hline
		\end{tabular}
	
	\end{table}

	\subsection{状态转移方程}
	状态转移方式很好求解,根据状态矩阵在实际情况中运算几步既可以得到一些方程式,让后将方程式中的参数换位形参符号即可。
	
	\noindent \textbf{第1阶段:}从n名学生中选取1名学生;显然每个学生都可能被单独抽中,即
	\begin{equation}
	pMax(1,1)=num(1),\ldots,pMax(1,n)=num(n)
	\label{1}
	\end{equation}
	
	
	\noindent \textbf{第2阶段:}从n名学生中选取2名学生;根据pMax(k,i)代表从n个学生中选取k名学生且最后一名的标号为i得
	\begin{equation}
	pMax(k,i)=0 , i<k
	\end{equation}即$pMax(2,1)=0$,表示此种方案不存在。
	\begin{equation}
	pMax(2,2)=max(pMax(1,1)*num(2),pMax(1,2)*num(2))
	\label{3}
	\end{equation}公式\ref{3}就已经告诉我们状态转移公式的普遍形式了。题目中要求选取相邻两个学生之间编号之差不超过d,将该要求导入公式同时将公式中的实参换为形参得
	\begin{equation}
	pMax(k,i)=max(pMax(k-1,i-1)*num(i),pMax(k-1,i-2)*num(i),\ldots,pMax(k-1,i-d)*num(i))
	\label{4}
	\end{equation}
	
	\noindent \textbf{特殊情况}:公式\ref{4}可以很好的应用在能力值都为正值的情况下,但实际情况能力值还可能存在负值,公式一旦遇到能力值为负值的学生就会跳过,但有时负负得正的取值可能比正正得正的取值还要大,因此还需要一个$pMin(k,i)$矩阵来考虑负值的情况。
		\begin{table}[htbp]
		\centering
		\caption{pMin矩阵}
		\begin{tabular}{|c|c|c|c|c|}
			\hline
			0& 1&2 & \ldots & n \\
			\hline
			1&pMin(1,1)&pMin(1,2)&\ldots&pMin(1,n)\\
			\hline
			2&pMin(2,1)&pMin(2,2)&\ldots&pMin(2,n)\\
			\hline
			\ldots&\ldots&\ldots&\ldots&\ldots\\
			\hline
			k&pMin(k,1) & pMin(k,2)&\ldots&pMin(k,n) \\
			\hline
		\end{tabular}
	\end{table}

	公式\ref{4}要改写为
	\begin{equation}
	\begin{split}
		pMax(k,i)=max(pMax(k-1,i-1)*num(i),pMin(k-1,i-1)*num(i),\ldots,\\
		pMax(k-1,i-d)*num(i),pMin(k-1,i-d)*num(i))
	\end{split}
	\label{5}
	\end{equation}
	
	\begin{equation}
	\begin{split}
	pMin(k,i)=min(pMax(k-1,i-1)*num(i),pMin(k-1,i-1)*num(i),\ldots,\\
	pMax(k-1,i-d)*num(i),pMin(k-1,i-d)*num(i))
	\end{split}
	\label{6}
	\end{equation}

	综合公式\ref{1}和公式\ref{5}和公式\ref{6},得到状态转移方程:
	
	\begin{equation*}
	\begin{cases}
		pMax(k,i)=pMin(k,i)num(i); k=1,i=1,2,\ldots,n \\
		\begin{cases}
		pMax(k,i)=max(pMax(k-1,i-j)*num(i),pMin(k-1,i-j)*num(i));\\
		pMin(k,i)=min(pMax(k-1,i-j)*num(i),pMin(k-1,i-j)*num(i));\\	
		\end{cases}k\neq 1,j=1,2,\ldots,d
	\end{cases}
	\end{equation*}
	\begin{align*}
	\end{align*}
	
	\section{C++代码}
	
	\begin{lstlisting}
	#include<iostream>
	using namespace std;
	#define MAX 100
	inline long long max(long long a,long long b){return a>b?a:b;}
	inline long long min(long long a,long long b){return a>b?b:a;}
	int main()
	{
		int N,K,D;
		int num[MAX]={0};
		long long pMax[MAX][MAX]={0};
		long long pMin[MAX][MAX]={0};
		cin >> N;
		for(int i=1;i<=N;i++)
		cin >> num[i];
		cin>>K>>D;
		for(int k=1;k<=K;k++)
		{
			for(int i=1;i<=N;i++)
			{
				if(k==1)
					pMax[k][i]=pMin[k][i]=num[i];
				else
				{
					for(int j=i-1;i-j<=D&&j>0;--j)
					{
					pMax[k][i]=max(pMax[k][i],max(pMax[k-1][j]*num[i],pMin[k-1][j]*num[i]));
					pMin[k][i]=min(pMin[k][i],min(pMax[k-1][j]*num[i],pMin[k-1][j]*num[i]));
					}
				}
			}
		}
		long long result = 0;
		for(int i=1;i<=N;i++)
			result = max(result,pMax[K][i]);
		cout << result;
		return 0;
	}	
	\end{lstlisting}
\end{document}