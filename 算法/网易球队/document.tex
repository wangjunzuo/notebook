\documentclass[a4paper]{article}
\usepackage[margin=1in]{geometry}
\usepackage{listings}
\usepackage{ctex}
\lstset{language=c++,tabsize=3}
\usepackage{amsmath}

\title{2018网易机试题球队}
\author{W.J.Z}
\date{2019.4.17}

\begin{document}
	\maketitle
	\section{题目}
	\textbf{问题:} 有三只球队,每只球队编号分别为球队1,球队2,球队3,这三只球队一共需要进行 n 场比赛。现在已经踢完了k场比赛,每场比赛不能打平,踢赢一场比赛得一分,输了不得分不减分。已知球队1和球队2的比分相差d1分,球队2和球队3的比分相差d2分,每场比赛可以任意选择两只队伍进行。求如果打完最后的 (n-k) 场比赛,有没有可能三只球队的分数打平。


	
	
	\textbf{输入描述:} 第一行包含一个数字 t (1 <= t <= 10)
	接下来的t行每行包括四个数字 n, k, d1, d2(1 <= n <= 10\^12; 0 <= k <= n, 0 <= d1, d2 <= k)
	
	
	\textbf{输出描述:} 每行的比分数据,最终三只球队若能够打平,则输出“yes”,否则输出“no”
	
	\section{思路}
	各大公司在线编程题大多都偏重于考察数学思维。首先从各个球队相差分数出发,三个队伍的分数可分为以下四种:
	
	\begin{enumerate}
		\item team1<team2<team3:设队伍的分数依次为m、m+d1、m+d1+d2
		\item team1>team2<team3:设队伍的分数依次为m+d1、m、m+d2
		\item team1<team2>team3: 设队伍的分数依次为m-d1、m、m-d2
		\item team1>team2>team3:队伍的分数依次为m、m-d1、m-d1-d2
	\end{enumerate}

	根据题目所给条件可以把三个球队的分数求出来,各队的分数为整数且必须大于等于0:
	\begin{enumerate}
		\item $m+m+d1+m+d1+d2 = k \rightarrow m=\frac{k-2d1-d2}{3}$
		\item $m+d1+m+d1+m+d2 = k \rightarrow m=\frac{k-d1-d2}{3}$
		\item $m-d1+m+m-d2 = k \rightarrow m=\frac{k+d1+d2}{3}$
		\item $m+m-d1+m-d1-d2 = k \rightarrow m=\frac{k+2d1+d2}{3}$
	\end{enumerate}
	
	球队打平手的条件公式:$(n/3-x_{1})+(n/3-x_{2})+(n/3-x_{3})=(n-k)$,$x_{1},x_{2},x_{3}$为每个球队的分数,化简后为$x_{1}+x_{2}+x_{3}=k$,看似好像没有得出任何有价值的条件,其实就隐含在公式中:$n/3-x_{1}$、$n/3-x_{2}$、$n/3-x_{3}$都是必须大于等于零且n能被3整除。
	
	\section{编程}
	
	\begin{enumerate}
		\item 条件1:球队的分数必须都大于等于0;编程时比较每种情况的最小分数即可:case1比较m,case2比较m,case3比较m-d1和m-d2,case4比较m-d1-d2.
		\item n必须能被三整除且$n/3$减去最大分数大于等于0;对于case1减去m+d1+d2,对于case2减去m+d1和m+d2,对于case3减去m,对于case4减去m。
	\end{enumerate}

	\begin{lstlisting}
	#include<iostream>
	#include<cmath>
	using namespace std;
	int main()
	{
		long long t,n,k,d1,d2,m,remaind;	
		cin >> t;
		for(int i=0;i<t;i++)
		{
			cin>>n>>k>>d1>>d2;
			m=(k-2*d1-d2)/3;
			remaind = (k-2*d1-d2)%3;
			if(m>=0&&remaind==0)
			{
				if(n%3==0&&(n/3-m-d1-d2)>=0)
				{
					cout<<"yes"<<endl;
					continue;
				}
			}
			m=(k-d1-d2)/3;
			remaind = (k-d1-d2)%3;
			if(m>=0&&remaind==0)
			{
				if(n%3==0&&(n/3-m-d1)>=0&&(n/3-m-d2)>=0)
				{
					cout<<"yes"<<endl;
					continue;
				}
			}
			m=(k+d1+d2)/3;
			remaind = (k+d1+d2)%3;
			if(m-d1>=0&&remaind==0&&m-d2>=0)
			{
				if(n%3==0&&(n/3-m)>=0)
				{
					cout<<"yes"<<endl;
					continue;
				}
			}
			m=(k+2*d1+d2)/3;
			remaind = (k+2*d1+d2)%3;
			if(m-d1-d2>=0&&remaind==0)
			{
				if(n%3==0&&(n/3-m)>=0)
				{
					cout<<"yes"<<endl;
					continue;
				}
			}
			cout<<"no"<<endl;
			}
			return 0;
	}
	\end{lstlisting}
	

\end{document}