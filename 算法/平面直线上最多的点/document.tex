\documentclass{article}
\usepackage{ctex}
\usepackage[top=1cm]{geometry}
\usepackage{listings}
\lstset{tabsize=5,language=C++}
\usepackage[ruled]{algorithm2e}

\title{二维平面直线上最多的点个数}
\author{W.J.Z}
\date{2019.4.10}

\begin{document}
	\maketitle
	\section{问题描述}
	Given n points on a 2D plane, find the maximum number of points that lie on the same straight line.
	\begin{lstlisting}
	* Definition for a point.
	* struct Point {
	*     int x;
	*     int y;
	*     Point() : x(0), y(0) {}
	*     Point(int a, int b) : x(a), y(b) {}
	* }; 
	\end{lstlisting}
	\section{问题分析}
	\subsection{问题解的所有情况}
	\noindent 第i个点与其他点有以下三种情况:
	\begin{enumerate}
		\item 重合:判断条件 $point[i].x==point[j].x$ \&\& $point[i].y==point[j].y$ 
		\item 无斜率(垂直): 判断条件  $point[i].x==point[j].x$
		\item 有斜率:判断条件 $k=y/x$
	\end{enumerate}
	\section{编程思路}
	针对重合情况,令变量cp记录与第i个点重合的点数;针对无斜率情况,令vp变量记录与第i个点垂直的点数;针对有斜率情况,由于斜率有很多,所有使用map<double,int>代表当斜率为k时的点数。
	\begin{lstlisting}
	int maxPoints(vector<Point> &points) {
		int size = points.size();
		if(size<=2 )
			return size;
		int vp,cp,cmax,res=0;
	
		for(int i=0;i<size;i++)
		{
			vp=0;cp=0;cmax = 0;
			map<double,int> grad;
			for(int j=i+1;j<size;j++)
			{
				double x = points[i].x - points[j].x;
				double y = points[i].y - points[j].y;
				if(x==0 && y==0)  //重合点
					cp++;
				else if(x==0)    //竖直方向
					vp++;
				else               //斜率
				{
					double k = y/x;
					if(grad.find(k)==grad.end())
						grad.insert(pair<double,int>(k,1));
					else
						grad[k]++;
				}
			}
			map<double,int>::iterator it = grad.begin();
			cmax = it->second;
			while(it!=grad.end())
			{
				cmax = max(cmax,it->second);
				it++;
			}
			res = max(max(cmax,vp)+cp+1,res);	
		}
		return res;
	}
	\end{lstlisting}
\end{document}