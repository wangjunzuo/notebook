\documentclass{article}
\usepackage{ctex}
\usepackage{listings}
\lstset{tabsize=3,language=C++,numbers=left,numbersep=2pt}
\usepackage[top=1cm]{geometry}
\title{大整数相乘}
\author{W.J.Z}
\date{2019.4.11}
\begin{document}
	\maketitle
	\section{Question}
	有两个用字符串表示的非常大的大整数,算出他们的乘积,也是用字符串表示。不能用系统自带的大整数类型。
	\section{method}
	\subsection{模拟乘法累加}
	将每一位相乘,相加的结果保存到同一个位置,到最后才计算进位.
	\begin{lstlisting}
	int main()
	{
		string str1,str2;
		cin >> str1>>str2;
		int result[MAX]={0};
		int num1[MAX]={0};
		int num2[MAX]={0};
		int size_num1 = str1.size();
		int size_num2 = str2.size();
		//注意是按照从右向左
		for(int i=0,j=size_num1-1;i<size_num1;i++,j--)
			num1[j]=str1[i]-'0';
		for(int i=0,j=size_num2-1;i<size_num2;i++,j--)
			num2[j]=str2[i]-'0';
		//相乘不进位
		for(int i=0;i<size_num1;i++)
			for(int j=0;j<size_num2;j++)
				result[i+j]=result[i+j]+num1[i]*num2[j];
		string str="";
		//进位
		for(int i=0;i<(size_num1+size_num2-1);i++)
		{
			result[i+1] = result[i+1]+result[i]/10;
			str+=to_string(result[i]%10);
		}
		//判断最后一位
		if(result[size_num1+size_num2-1]!=0)
			str+=to_string(result[size_num1+size_num2-1]);
		//逆序输出
		for(int i=str.size()-1;i>=0;i--)
		cout<<str[i];
	
		return 0;
	}

	\end{lstlisting}
		
\end{document}