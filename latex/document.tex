\documentclass{article}
\usepackage{ctex}
\usepackage{enumerate}
\usepackage{amsmath}
\usepackage{listings}
\usepackage{biblatex}
\addbibresource{refer.bib}
\title{{论文格式}\\
	{\large ---以中文论文为例}}
\author{W.J.Z}
\date{}
\begin{document}
	
\maketitle
\begin{abstract}
	本文总结论文写作中了textstudio软件中文设置、论文结构、插图、表格、公式和其他一些设置,文章以后查阅之用,因作者水平有限,如有不足之处请多多指教,后续将继续完善内容。
\end{abstract}
%\textbf{摘要:}

\section{中文设置}
	\begin{enumerate}[步骤1:]
		\item Options(选项) 
		\item 软件右下角选为 UTF-8
		\item $\backslash$usepackage\{ctex\}
	\end{enumerate}

\section{论文结构}
	\begin{enumerate}[结构1:]
		\item 	一级标题:$\backslash$section\{一级标题题目\}
		\item   二级标题:$\backslash$subsection\{二级标题题目\}
		\item  三级标题:$\backslash$subsubsection\{三级标题题目\}
	\end{enumerate}

\section{插图}
	图片和表格一般都设置为浮动体,而图片一般都为几排几列格式,下面演示这种方法设置:
	\begin{enumerate}[步骤1:]
		\item $\backslash$usepackage\{graphicx\}\\
				 $\backslash$usepackage\{subfigure\}\\
				  $\backslash$graphicspath\{\{图片所在的路径\}\}
		\item 使用minipage命令显示所想要的格式\\
				$\backslash$begin\{figure\}[htbp] \% h:here t:top b:bottom p:page  \\
			    $\backslash$centering \%居中显示 \\
			    $\backslash$begin\{minipage\}[b]\{0.24$\backslash$textwidth\} \%该小页面占文本宽度的\%24 \\   
			    $\backslash$includegraphics[width=$\backslash$textwidth][a]  \%加载图片名为a的图片 \\
			    $\backslash$caption *\{图片1\} \% 对分吐取标题不进行标号 \\
			    $\backslash$end\{minipage\}  \\
			    \ldots \\
			    $\backslash$caption *\{图片\} \% 对整块图取标题进行标号 \\
			   $\backslash$label\{d\}  \%方便引用 \\
			   $\backslash$\{figure\}
	\end{enumerate}
	使用eps文件格式
	\begin{lstlisting}
	\usepackage{epstopdf}
	\end{lstlisting}
\section{表格}
   \begin{enumerate}[步骤1:]
	\item	$\backslash$begin\{table\}  \\
			$\backslash$centering \\
			$\backslash$caption\{表格\}  \% 对表格取标题 \\
			$\backslash$label\{t\}   \%设置引用名称 \\
			$\backslash$begin\{tabular\}\{|c c |\}  \% c:居中 |:画竖线 \\
			$\backslash$hline \%画横线 \\
			姓名 \& 年龄  \% 数据以\&进行分割 \\
			\ldots   \\
			$\backslash$end\{tabular\} \\
			$\backslash$end\{table\}
	\end{enumerate}
	caption位于表格左上方
	\begin{lstlisting}
	\captionsetup[table]{
	labelsep=newline,%换行
	singlelinecheck=false,%居左
	}
	\end{lstlisting}
\section{公式}

	\begin{enumerate}[格式1:]
		\item 	$\backslash$begin\{equation\}	\%单行对齐 自动编号 \\
				\ldots  \\
				$\backslash$end\{equation\}
				
		\item $\backslash$begin\{align*\}  \\
		   \&  \quad \%以\&后面的符号为对齐标准  \\
		$\backslash$end\{align*\}
		
	\end{enumerate}
 条件函数:
	\begin{lstlisting}
	\begin{equation}
	D(x)=\begin{cases}
	1, & \text{}\\
	0, &  \text{}
	\end{cases}
	\end{equation}
	\end{lstlisting}
\section{代码}
	\begin{enumerate}
	\item	$\backslash$usepackage\{listing\} \\
		   $\backslash$lstset\{language=python,tabsize=4\}  \%指定编程语言\\
		   	$\backslash$begin\{lstlisting\}\\
		   		$\backslash$end\{lstlisting\}
	\end{enumerate}
\section{算法}
\begin{lstlisting}
	\usepackage[repeat]{algorithm2e}
	\begin{algorithm}[H]
	\caption{}  %算法名字
	\LinesNumbered  %显示行号
	\Kwln{}  %输入参数
	\Kwout{} %输出
	\repeat{}{}
	\For{}{}
	\end{algorithm}
\end{lstlisting}
\section{改变某行颜色}
\begin{lstlisting}
	\usepackage{color}   %加载支持包
	{\color{red}{文本}}
\end{lstlisting}
\section{论文引用}
\begin{lstlisting}
#修改配置
option->configure texstudio->build->bibligraphy tool->Biber
#加载支持包
\usepackage{biblatex}
#创建*.bib文件 在百度学术复制bib格式引用
\addbibresource{ref.bib}
#进行引用
\cite{}
#显示引用
\printbiblography
\end{lstlisting}
\section{转置}

\begin{lstlisting}
	\usepackage{rotating} \begin{sidewaystable}
	\centering
	\end{sidewaystable}
\end{lstlisting}
\subsection{标题}
\begin{lstlisting}
	\usepackage[labelsep=space]{caption}
	\captionsetup[table]{
	labelsep=newline,%换行
	singlelinecheck=false,%居左
	}
\end{lstlisting}
\section{其他}
	\begin{enumerate}[引用:]
		\item $\backslash$ref\{引用名字\}
	\end{enumerate}
	\begin{lstlisting}
	\noindent  取消改行缩进
	\mathop(min)_{i,j} 下标
	指定纸张
	\usepackage[a4paper, left = 2cm, right = 2cm]{geometry} 
	添加页眉
	\usepackage{fancyhdr}
	\pagestyle{fancy}
	\lhead{}
	\chead{}
	% bfseries
	\chead{\bfseries W.J.Z copyright reserved}
	
	#字母上方
	\hat{}
	\end{lstlisting}
\end{document}