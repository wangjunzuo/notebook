\documentclass{article}
\usepackage{ctex}
\usepackage{listings}
\title{C++面向对象编程}
\author{W.J.Z}
\date{}
\begin{document}
	\maketitle
	\section{C++类}
	\begin{lstlisting}[language=C++,tabsize=2]
	calss Sales_data add{
		friend Sales_data add(const Sales_data&,const Sales_data&);
		public:
			Sales_data() = dafault;
			Sales_data()(const std::string &s,unsigned n,double p):
						bookNo(s),units_solds_sold(n),revenue(p*n){}
			Sales_data(const std::string&s):bookNo(s){}
			Sales_data(std::istream&);
			std::string isbn() const {return bookNo;}
			Sales_data &combine(const Sales_data&);
			
		privete:
			std::string bookNo;
			unsigned unites_sold = 0;
			double revenue = 0.0;
	};
	封装实现了类的接口和实现的分离,封装后的类隐藏了它的实现细节。
	编译器处理类分为两步:首先编译成员的声明,然后才轮到成员函数体。
	calss和struct定义类唯一区别就是默认访问权限不同。
	成员函数通过一个名为this的额外的隐式参数来访问调用它的那个对象。
	\end{lstlisting}
	\section{拷贝控制}
	\section{重载运算与类型转换}
	\section{面向对象编程设计}
	\section{模板与泛型编程}
\end{document}