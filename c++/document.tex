\documentclass{article}
\usepackage{ctex}
\usepackage{listings}
\usepackage{enumerate}
\lstset{language=C,xleftmargin=-3em,tabsize=4}
\title{c++总结}
\author{W.J.Z}
\date{}
\begin{document}
\maketitle
\section{变量和基本类型}
	\noindent 1、基本内置类型:
	
	C++基本数据类型为算术类型(布尔、字符、整型、浮点型)和空类型。\\
	
	\begin{tabular}{|c|c|}
		\hline
		类型 & 最小尺寸 \\
		\hline
		bool & 未知 \\
		char & 8位 \\
		wchar\_t & 16位 \\
		char16\_t & 16位\\
		char32\_t & 32位 \\
		int & 16位 \\
		long & 16位 \\
		long long & 32位\\
		float & 6位有效数字\\
		double & 10位有效数字 \\
		long double & 10位有效数字\\
		\hline
	\end{tabular}	

	\begin{enumerate}[技巧1:]
		\item 执行浮点数运算选double
		\item 数值不为负时选unsigned
		\item 整型运算选择int和long long
	\end{enumerate}
		
	\noindent 2、变量声明与定义
	
	变量初始化时创建变量并赋予一个初始值,赋值为擦除变量当前值赋予一个新值,推荐初始化使用$int \quad sum(0);$方式。
	
	为了配合c++分离式编译,使变量为整个程序所知,使用extern关键字对变量进行声明。变量只能被定义一次,但可以被多次声明。
	\\
	
	\noindent 3、引用和指针的区别
	
	引用为引用对象创建了一个别名,指针则是创建了对象。指针无需在创建时进行赋初值,引用必须制定引用对象。指针可以在整个生命周期指向不同的对象,而应用有且只有一个。
	\begin{lstlisting}
	int *p = &a; %复合类型写法一
	int* p = &a; %复合类型写法二
	const int a =10; %a值不能改变
	int *const p = &b;%不能改变p的值
	%*放在const前面说明指针是一个常量
	const int* p = &a;%a值不能改变
	const int* cont p = &a;%a和p值都不能改变
	\end{lstlisting}
	
	\noindent 4、头文件
	
	$\#define$定义预处理变量,$ifdef 和 ifndef$检查预处理变量是否定义,如果检查为真,执行后续操作直到遇见$\#endif$。
	
	\section{面向对象的基本特性}
	\begin{enumerate}
		\item 类和对象的申明\\
		\item 构造函数
		\item 析构函数
		\item 内联成员函数
		\item 复制构造函数
		\item 类的组合	
		\item 结构体
		\item 联合体
		\item 枚举类	
	\end{enumerate}
	\section{数据共享与保护}
	\begin{enumerate}
		\item 静态数据成员
		\item 静态函数成员
		\item 友元函数和友元类
		\item 常对象、常成员、常引用
		\item 多文件结构
		\item 预编译命令
	\end{enumerate}
	\section{继承和派生}
	\begin{enumerate}
		\item 单继承、多继承、继承方式
		\item 基类与派生类转换	
	\end{enumerate}
	\section{面向对象编程}
	对于某些函数,基类希望它的派生类各自定义适合自身的版本,此时基类就将这些函数声明成虚函数,派生类必须在其内部对其所有重新定义的虚函数进行声明,任何构造函数之外的非静态函数函数都可以是虚函数,如果基类把一个函数声明为虚函数,则该函数在派生类中隐式地也是虚函数,派生类对象不能直接初始化从基类继承而来的成员(每个类控制自己成员的初始化过程),首先初始化基类的部分,然后按照声明的顺序依次初始化派生类的成员。
	
	如果基类定义了一个静态成员,则在整个继承体系中只存在该成员的唯一定义。C++11新标准中可以使用override关键字来说明派生类中的虚函数,将共同基类摄这位虚基类,这时从不同的路径继承过来的同名数据成员在内存中就有一个副本,同一个函数名也只有一个映射。
	
\end{document}