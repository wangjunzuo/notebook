\documentclass{article}
\usepackage{ctex}
\usepackage[a4paper, left = 3cm, right = 3cm,top=3cm]{geometry}

\usepackage{fancyhdr}
\pagestyle{fancy}
\lhead{}
\chead{}
% bfseries
\chead{\bfseries W.J.Z copyright reserved}
%\lfoot{From: K. Grant}
%\cfoot{To: Dean A. Smith}
%\rfoot{\thepage}
%\renewcommand{\headrulewidth}{0.4pt}
%\renewcommand{\footrulewidth}{0.4pt}

\title{统计学习方法总结}
\author{W.J.Z}
\date{}


\begin{document}
	\maketitle
	\textbf{\small 本文将感知机、K近邻法、朴素贝叶斯、决策树、逻辑斯蒂回归和最大熵模型、支持向量机、提升方法、EM算法、隐马尔可夫模型的特点总结概括在表1中。}
	\begin{table}[h]
		\caption{9种统计学习方法特点概况总结}
		\centering
		\begin{tabular}{|p{1.7cm}|p{1.7cm}|p{1.7cm}|p{1.7cm}|p{1.7cm}|p{1.7cm}|p{1.7cm}|}
			\hline
			方法 & 适用问题 & 模型特点 & 模型类型 & 学习策略 & 学习的损失函数 & 学习算法 \\
			\hline
			感知机 & 二类分类 & 分离超平面 & 判别模型 & 极小化误分点到超平面距离 & 误分点到超平面距离 & 随机梯度下降 \\
			\hline
			k近邻法 & 多类分类,回归 & 特征空间,样本点& 判别模型& & & \\
			\hline
			朴素贝叶斯法 & 多类分类& 特征与类别的联合概率分布,条件独立假设 & 生成模型 & 极大似然估计,极大后验概率估计&对数似然损失 & 概率计算公式,EM算法\\
			\hline
			决策树 & 多类分类,回归 &分类树、回归树 & 判别模型 & 正则化的极大似然估计 & 对数似然损失 & 特征选择、生成、剪枝\\
			\hline
			逻辑斯蒂回归与最大熵模型 & 多类分类 & 特征条件下类别的条件概率分布,对数线性模型 & 判别模型 & 极大似然估计、正则化的极大似然估计& 逻辑斯蒂损失 & 梯度下降,拟牛顿法 \\
			\hline
			支持向量机 & 二类分类 & 分离超平面,核技巧 & 判别模型 & 极小化正则化合页损失,软间隔最大化 & 合页损失 &序列最小最优化算法 \\
			\hline 
			提升方法 & 二类分类 & 若分类器的线性组合 & 判别模型& 极小化加法模型的指数损失 & 指数损失 & 前向分布加法算法 \\
			\hline
			EM算法 & 概率模型参数估计 & 含隐变量概率模型 & &极大似然估计,极大后验概率估计&对数似然估计&迭代算法 \\
			\hline
			隐马尔可夫模型 & 标注 & 观测序列与状态序列的联合概率分布模型 & 生成模型 & 极大似然估计、极大后验概率估计&对数似然损失 & 概率计算公式、EM算法\\
			\hline
		\end{tabular}
	\end{table}
\end{document}