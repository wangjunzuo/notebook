\documentclass{article}
\usepackage{ctex}
\usepackage{listings}
\usepackage[ruled]{algorithm2e}
\usepackage[top=1cm]{geometry}
\lstset{xleftmargin=-2em,tabsize=4,}
\title{K近邻}
\author{W.J.Z}
\date{}

\begin{document}
	\maketitle
	\textbf{摘要:}本文重点介绍了K近邻距离度量、kd树生成、kd树最近邻搜索。算法和标准算法有所差别但效果都是一样的,理解难点在于切分时递归和搜索时半径与维度距离大小比较,自己动手模拟过程可以帮助理解。
	\section{近邻模型}
	K近邻算法:给定一个训练数据集,对新的输入实例,在训练数据集中找到与该算法最邻近的K个实例。模型由距离度量、K值、分类决策规定决定。
	
	\noindent 欧式距离:
	\begin{equation}
	L_{p}\left ( x_{i},x_{j} \right )=\left ( \sum_{l=1}^{n}\left | x_{i}^{l}-x_{j}^{l} \right |^{2} \right )^{\frac{1}{2}}
	\end{equation}
	
	\noindent 曼哈顿距离:
	\begin{equation}
	L_{p}\left ( x_{i},x_{j} \right )= \sum_{l=1}^{n}\left | x_{i}^{l}-x_{j}^{l} \right | 
	\end{equation}
	
	\noindent  切比雪夫距离 :
	\begin{equation}
	L_{p}\left ( x_{i},x_{j} \right )= \max\limits_{l}\left |x_{i}^{l}-x_{j}^{l}  \right |
	\end{equation}
	\section{kd树构造}
	k近邻法最简单的实现方式为线性扫描,当训练集很大时,计算非常耗时,为提高搜索效率,使用特殊存储结构储存训练数据,kd树就是其中一种方法。
	
	\begin{algorithm}[H]
		\caption{构造平衡二叉树}
		\LinesNumbered	
		\KwIn{训练数据集}
		\KwOut{kd树}
		构造根节点,选择实例$(depth \% data.length)$维度的中位数为切分点,将对应的区域分成两个子区域。左边区域节点对应坐标小于且分点,右边区域节点大于切分点。
		
		记录该深度落在切分超平面实例的下标,重复步骤一操作,直到两个子区域没有实例存在为止。
		
	\end{algorithm}
	
	\noindent kd关键算法代码:
	\begin{lstlisting}{language=c++}
	#kd树构造利用二叉树递归
	#二叉树节点
	struct Node{
		int index; #记录落在超平面实例的下标
		int axis; #记录该节点所在的维度
		Node *left; #左区域
		Node *right; #右区域
		Node():index(-1),axis(-1){left=right=nullptr}
	};
	
	/**建树
	*indices记录每一个落在超平面实例的下标,二叉树节点储存的是下标
	*不是实例,indices初始值为{0,1,2,3...n},n为数据集的大小,len为
	*剩余搜索长度,depth为二叉树深度
	*/
	Node* buildTree(int* indices,int len,int depth)
	{
		if len <=0 
			return nullptr;
		//依次递增维度,length为特征向量长度
		const int axis = depth % length;
		//获取中位数
		const int mid = len / 2;
		#依照数据集选定的维度数据进行indices排序获取中位数
		for(i=0;i<len,i++)
			temp[i] = *((indices+i)[axis]);
		qQsort(temp,indeices,0,len(data));
		/**qQsort函数等同于使用STL标准库的函数如下:
		std::nth_element(indices, indices + mid, indices + len
		, [&](int lhs, int rhs){
		return data[lhs][axis] < data[rhs][axis];
		});
		*/
		Node * node = new Node();
		node->index = indices[mid];
		node->axis = axis;
		node->left = buildTree(indices,mid,depth+1);
		node->right = buildTree(indices+mid+1,len-mid-1,depth+1);
		#indices+mid+1移动指针
		return node;
	}	
	void qQsort(int temp[],int* p,int low,int high)
	{
		if(low >= high)
			return ;
		int first = low;
		int last = high;
		int key = temp[first];
		while(first<last)
		{
			while(first < last && temp[last] >= key)
				--last;
			temp[first]=temp[last];
			while(first < last && temp[first] <= key)
				++first;
			temp[last] = temp[first];
		}
		a[first] = key;
		//对indices进行调整排序
		a =  p[first];
		p[first] = p[low];
		p[low] = a;
		qQsort(temp,low,first-1);
		qQsort(temp,first+1,high);
	}
	\end{lstlisting}
	

	\section{kd树最近邻搜索}
	
		\begin{algorithm}
		\caption{kd树最近邻搜索}
		\LinesNumbered
		\KwIn{kd树,目标点$x$}
		\KwOut{$x$的最近邻}	
		从根节点出发,递归向下访问kd树,若目标点当前维的坐标小于且分点的坐标,移动到左节点,否则移动到右节点,直到子节点为叶节点未知。同时计算目标点与个切分点的欧式距离,记录最小距离和最近节点。
		
		递归向上回退,计算当前点的维度与目标点之间的距离,如果小于最小距离,说明当前节点另外一个子节点可能存在离目标节点更近的节点,进入该子区域,重复步骤一。
		
		当退回根节点,搜索结束,最后的最近节点为目标节点的最近邻节点。
	\end{algorithm}
	
	\begin{lstlisting}
	void search(const int& query,const Node* node,int* result,
	double* minDist)
	{
		if(node==nullptr)
			return;
		//获取当前节点数据
		const int& train = data[node->index];
		//计算当前节点与目标节点的欧式距离
		//distance()计算欧式距离函数
		const double dist = distance(query,train);
		//记录最近距离和坐标
		if(dist < *minDist)
		{
			*minDist = dist;
			*result = node->index;
		}
		const int axis = node->axis;
		//进行当前节点维度比较,进而选择左右树
		const int dir = query[axis] < train[axis]?0:1;
		if(dir == 0)
			search(query,node->left,result,minDist);
		else
			search(query,node->left,result,minDist);
		//计算当前节点维度与目标节点的距离
		const double diff = fabs(query[axis]-train[axis]);
		//如果在半径之内,说明当前节点子区域可能存在更近的点
		if(diff < *miDist)
		{
			if(!dir)
				search(query,node->right,result,minDist);
			else
				search(query,node->left,result,minDist);
		}
	}
	\end{lstlisting}
	\textbf{参考资料:}https://www.cnblogs.com/earendil/p/8135074.html\\
	\textbf{大佬代码参考:}https://github.com/benjones/kdTree
	
	\section{性能度量}
	聚类性能度量大致分为两类:一类将聚类结果与某个参考模型进行比较,成为外部指标;一类直接考察聚类结果而不利用任何参考模型,称为内部指标。
	\subsection{外部指标}
	
	假设通过聚类给出的簇划分为$C={C_{1},C_{2},C_{3},\ldots,C_k}$,参考模型给出的簇划分为$C^{*}={C_{1}^{*},\ldots,C_{s}^{*}}$,令$\lambda $和$\lambda ^{*}$分别表示与$C$和$C^{*}$对应的簇标记向量。
	\begin{equation}
	a=\left | SS \right |;SS=\left \{ \left ( x_{i},y_{j} \right )|\lambda _{i}=\lambda _{j},\lambda _{i}^{*}=\lambda _{j}^{*},i< j \right \}
	\end{equation}
	\begin{equation}
	b=\left | SD \right |;SS=\left \{ \left ( x_{i},y_{j} \right )|\lambda _{i}=\lambda _{j},\lambda _{i}^{*}\neq \lambda _{j}^{*},i< j \right \}
	\end{equation}
	\begin{eqnarray}
	c=\left | DS \right |;SS=\left \{ \left ( x_{i},y_{j} \right )|\lambda _{i}\neq \lambda _{j},\lambda _{i}^{*}=  \lambda _{j}^{*},i< j \right \}\\
	d=\left | DD \right |;SS=\left \{ \left ( x_{i},y_{j} \right )|\lambda _{i}\neq \lambda _{j},\lambda _{i}^{*}\neq   \lambda _{j}^{*},i< j \right \}
	\end{eqnarray}
	\begin{enumerate}
		\item Jaccard系数
		\begin{equation}
		JC=\frac{a}{a+b+c}
		\end{equation}
	\item FM指数
	\begin{equation}
	FMI=\sqrt{\frac{a}{a+b}\cdot \frac{a}{a+c}}
	\end{equation}
	\item Rand指数
	\begin{equation}
	RI=\frac{2\left ( a+d \right )}{m\left ( m-1 \right )}
	\end{equation}
	\end{enumerate}
	上述性能度量的结果均在$[0-1]$区间,值越大越好。
	
	\subsection{内部指标}
	\begin{eqnarray}
	avg(C)=\frac{2}{\left | C \right |\left ( \left | C \right |-1 \right )}\sum _{1\leq i\leq j\leq \left | C \right |}dist(x_{i},x_{j})\\
	diam(C)=max_{1\leq i\leq j\leq \left | C \right |}dist(x_{i},x_{j})\\
	d_{min}\left ( C_{i},C_{j} \right )=min_{x_{i}\epsilon C_{i},x_{j}\epsilon C_{j}}dist(x_{i},y_{j})\\
	d_{cen}(C_{i},C_{j})=dist(\mu _{i},\mu _{j})	
	\end{eqnarray}
	$dist()$表示两个样本之间的距离,$\mu $代表簇C的中心点,$avg(C)$对应簇C内样本间的平均距离,diam(C)对应于簇C内样本间的最远距离,$d_{min}(C_{i},C_{j}$对应簇$C_{i}$与簇$C_{j}$最近样本间的距离,$d_{cen}(C_{i},C_{j})$对应于簇$C_{i}$与簇$C_{j}$中心点的距离。
	
	\begin{enumerate}
		\item DB指数
		\begin{eqnarray}
		DBI=\frac{1}{k}\sum_{k}^{i=1}max_{j\neq i}\left ( \frac{avg(C_{i})+avg(C_{j})}{d_{cen}(\mu _{i},\mu _{j})} \right )
		\end{eqnarray}
		\item Dunn指数
		\begin{equation}
		DI=min_{1\leq i\leq k}\left \{ min_{j\neq i}\left ( \frac{d_{min}\left ( C_{i},C_{j} \right )}{max_{1\leq l\leqslant k}diam(C_{l})} \right ) \right \}
		\end{equation}
	\end{enumerate}
	DBI值越小越好,DI值越大越好。
\end{document}