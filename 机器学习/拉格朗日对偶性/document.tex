\documentclass{article}
\usepackage{ctex}
\usepackage{amsmath}

\title{拉格朗日对偶性}
\author{W.J.Z}
\date{}

\begin{document}
	\maketitle
	拉格朗日对偶性通常解决约束最优化问题,通过解对偶问题来求出原始问题的解。
	\section{原始问题}
	设$f(x),c_{i}(x),h_{j}(x)在R^{n}$上的连续可微函数,约束最优化问题为:
	\begin{equation}
	\mathop{min}_{x\varepsilon R^{n}}f\left ( x \right )
	\end{equation}
	\begin{equation}
	s.t. \quad c_{i}(x)\leq 0, i =1,2,\ldots,k
	\label{2}
	\end{equation}
	\begin{equation}
	 \quad c_{j}\leq 0, j =1,2,\ldots,l
	 \label{3}
	\end{equation}
	引入拉格朗日函数
	\begin{equation}
	L\left ( x,\alpha ,\beta  \right )=f\left ( x \right )+\sum_{i=1}^{k}\alpha _{i}c_{i}\left ( x \right )+\sum_{j=1}^{l}\beta _{i}h_{j}\left ( x \right )
	\label{4}
	\end{equation}
	$\alpha_{i}\geq 0$考虑$x$的函数
	\begin{equation}
	\Theta _{P}(x)=\mathop{max}_{\alpha ,\beta ;\alpha _{i}\geq 0}L\left ( x,\alpha ,\beta  \right )
	\label{5}
	\end{equation}
	P表示原始问题。\\
	假设给定某个$x$,如果$x$违反原始问题的约束条件,使$c_{i}(x)>0$或$h_{j}\neq 0$则:
	\begin{equation}
	\Theta _{P}(x)=\mathop{max}_{\alpha ,\beta ;\alpha _{i}\geq 0}L\left ( x,\alpha ,\beta  \right )=+\infty 
	\end{equation}
	如果$x$满足约束条件公式\ref{2}和\ref{3}的话,由公式\ref{4}和\ref{5}可知:
	\begin{align*}
	\Theta _{P}\left ( x \right )=\begin{cases}
	f(x) & \text{$x$满足原始问题约束}\\
	+\infty & \text{其他}
	\end{cases}
	\end{align*}
	所以考虑极小化问题
	\begin{equation}
	\mathop{min}_{x}\Theta _{P}\left ( x \right )=\mathop{min}_{x}\mathop{max}_{\alpha ,\beta ;\alpha _{i}\geq 0}L\left ( x,\alpha ,\beta  \right )
	\end{equation}
	它与原始问题最优解等价,即它们有相同的解。定义原始问题的最优解为:
	\begin{equation}
	p^{\ast }=\mathop{min}_{x}\Theta _{P}\left ( x \right )
	\end{equation}
	\section{对偶问题}
	广义拉格朗日函数的极大极小问题:
	\begin{equation}
	\mathop{max}_{\alpha ,\beta ;\alpha _{i}\geq 0}\theta_{D}\left (  \alpha ,\beta \right )=\mathop{max}_{\alpha ,\beta ;\alpha _{i}\geq 0}\mathop{min}_{x}L\left ( x,\alpha ,\beta  \right )
	\end{equation}
	将广义拉格朗日函数的极大极小问题表示为约束最优化问题:
	\begin{equation}
	\mathop{max}_{\alpha ,\beta }\theta_{D}\left (  \alpha ,\beta \right )=\mathop{max}_{\alpha ,\beta }\mathop{min}_{x}L\left ( x,\alpha ,\beta  \right )
	\end{equation}
	\begin{equation}
	s.t. \quad \alpha _{i}\geq 0 , \quad i=1,2,\ldots,k
	\end{equation}
	定义对偶问题的最优解:
	\begin{equation}
	d^{*}=\mathop{max}_{\alpha ,\beta ;\alpha _{i}\geq 0}\theta _{D}\left ( \alpha ,\beta  \right )
	\end{equation}
	\section{原始问题和对偶问题的关系}
	\noindent 定理1:
	\begin{equation}
	d^{*}=\mathop{max}_{\alpha ,\beta }\mathop{min}_{x}L\left ( x,\alpha ,\beta  \right )\leq =\mathop{min}_{x}\mathop{max}_{\alpha ,\beta ;\alpha _{i}\geq 0}L\left ( x,\alpha ,\beta  \right )=p^{*}
	\end{equation}
	定理2:假设$f(x)$函数和$c_{i}(x)$是凸函数,$h_{j}(x)$是放射函数,并且假设不等式约束$c_{j}(x)$是严格可行的,则存在$x^{*},\alpha ^{*},\beta ^{*}$,使得$x^{*}$是原始问题的解,$\alpha ^{*},\beta ^{*}$释对偶问题的解,并且:
	\begin{equation}
	p^{*}=d^{*}=L\left \{ x^{*},\alpha ^{*},\beta ^{*} \right \}
	\end{equation}
	定理3:定理2的充分必要条件为KKT条件:
	\begin{equation}
	\bigtriangledown _{x}L\left ( x^{*},\alpha ^{*},\beta ^{*} \right )=0
	\end{equation}
	\begin{equation}
	\alpha _{i}^{*}c_{i}\left ( x^{*} \right )=0,\quad i=1,2,...,k
	\end{equation}
	\begin{equation}
	c_{j}\left ( x^{*} \right )\leq 0 ,\quad i=1,2,...,k
	\end{equation}
	\begin{equation}
	\alpha _{i}^{*}\geq 0, \quad i=1,2,\ldots,k
	\end{equation}
	\begin{equation}
	h_{j}\left ( x^{*} \right )=0 \quad j=1,2,\ldots,l
	\end{equation}
\end{document}