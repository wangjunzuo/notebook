\documentclass{report}
\usepackage{ctex}
\usepackage{amsmath}
\usepackage[ruled]{algorithm2e}
\title{无约束最优化问题求解方法}
\author{W.J.Z}
\date{}

\begin{document}
	\maketitle
	\section{梯度下降法}
	假设$f(x)$有一阶偏导数,若第K次迭代值为$x^{k}$,则可将$f(x)$在$x^{k}$附近展开一阶泰勒展开式。
	$$f\left ( x \right )=f\left ( x^{k} \right )+\bigtriangledown f\left ( x^{k} \right )$$
	$\bigtriangledown f\left ( x^{k} \right )$为$f(x)$在$x^{k}$的梯度。
	
	第$k+1$逸代值为:$$x^{k+1}\leftarrow x^{k}+\lambda _{k}p_{k}$$
	$p_{k}$为搜索方向,$p_{k}=-\bigtriangledown f\left ( x^{k} \right )$,$\lambda _{k}$为搜索步长,使得:
	$$f\left ( x^{k}+\lambda _{k}p_{k} \right )=\mathop{min}_{\lambda \geq 0}f\left ( x^{k}+\lambda p_{k} \right )$$
	\begin{algorithm}[h]
		\caption{梯度下降法}
		\LinesNumbered
		\KwIn{目标函数$f(x)$,梯度函数$\bigtriangledown f\left ( x^{k} \right )$,计算精度$\varepsilon $}
		\KwOut{$f(x)$的极小值点}
		取初始值$x^{0}$,置$k=0$
		
		计算$f(x^{k})$
		
		计算梯度$\bigtriangledown f\left ( x^{k} \right )$,当$\left \| \bigtriangledown f\left ( x^{k} \right ) \right \|< \varepsilon $停止迭代,令极小点为$x^{k}$;否则令$p_{k}=-\bigtriangledown f\left ( x^{k} \right )$,求$\lambda _{k}$,使$$f\left ( x^{k}+\lambda _{k}p_{k} \right )=\mathop{min}_{\lambda \geq 0}f\left ( x^{k}+\lambda p_{k} \right )$$
		
		$x^{k+1}= x^{k}+\lambda _{k}p_{k}$计算$f(x^{k+1})$,当$\left \| f\left ( x^{k+1} \right ) -f\left ( x^{k} \right )\right \|< \varepsilon $或$\left \| x^{k+1}-x^{k} \right \|< \varepsilon $时,停止迭代令极小值点为$x^{k+1}$
		
		否则,$k=k+1$,转到步骤3
	\end{algorithm}
	\section{牛顿法和BFGS}
	牛顿法时迭代算法,每一步需要求解目标函数的海塞矩阵的逆矩阵,计算比较复杂。拟牛顿法通过正定矩阵近似海塞矩阵,简化这一计算过程.
	\subsection{牛顿法}
	假设$f(x)$有二阶连续偏导数,如第k次迭代值为$x^{k}$,则可在$f(x)$在$x^{k}$附近展开二阶泰勒公式:
	$$f\left ( x \right )=f\left ( x^{k} \right )+\bigtriangledown \left ( x-x^{k} \right )+\frac{1}{2}\left ( x-x^{k} \right )^{T}H\left ( x^{k} \right )\left ( x-x^{k} \right )$$
	
	$\bigtriangledown f\left ( x^{k} \right )$为$f(x)$在$x^{k}$的梯度,$H(x^{k})$是$f(x)$的海塞矩阵:
	$$H\left ( x \right )=\left [ \frac{\partial  ^{2}f}{\partial x_{i}\partial x_{j}} \right ]$$
	函数$f(x)$有极值的必要条件是在极值点处一阶导数为0,即梯度向量为0,当海塞矩阵是正定矩阵时$(X^{-1}H\left ( x \right )X)>0$,函数的极值为极小值。\\
	
	假设$x^{k+1}$满足$\bigtriangledown f\left ( x^{k+1} \right )=0$,泰勒公式求导并将$x^{k+1}$代入得:
	$$\bigtriangledown f\left ( x^{k} \right ) + H_{k}\left ( x^{\left ( k+1 \right )-}x^{\left ( k \right )} \right )=0$$
	因此$$x^{\left ( k+1 \right )}=x^{\left ( k \right )}-H_{k}^{-1}\bigtriangledown f\left ( x^{k} \right )$$
	令$P_{k}=-H_{k}^{-1}\bigtriangledown f\left ( x^{k} \right )$.
	\begin{algorithm}
		\caption{牛顿法}
		\LinesNumbered
		\KwIn{目标函数$f(x)$,梯度函数$ \bigtriangledown f(x)$,海塞矩阵$H(x)$和精度$\varepsilon $}
		\KwOut{函数极小值点}
		取初始值$x^{0}$,置$k=0$
		
		计算$ \bigtriangledown f\left ( x^{k} \right ) $
		
		若$\left \| \bigtriangledown f\left ( x^{k} \right ) \right \|< \varepsilon $,则停止计算,得到近似解$x^{k}$.
		
		计算$H(x^{k})$,求$$P_{k}=-H_{k}^{-1}\bigtriangledown f\left ( x^{k} \right )$$
		
		置$x^{(k+1)}=x^{k}+p_{k}$
		
		置$k=k+1$,转到步骤2
	\end{algorithm}
	\newpage
	拟牛顿算法使用一个n阶矩阵来代替海塞矩阵,来降低因海塞矩阵导致计算度复杂的问题。
	
	\begin{algorithm}[h]
		\caption{BFGS算法}
		\LinesNumbered
		\KwIn{目标函数$f(x)$,梯度函数$ \bigtriangledown f(x)$,精度$\varepsilon $}
		\KwOut{函数极小值点}
		初始点$x^{(0)}$,取$B_{0}$为正定对称矩阵,置K=0.
		
		计算$\left \| \bigtriangledown f\left ( x^{k} \right ) \right \|< \varepsilon $,则停止计算,得到近似解$x^{k}$.
		
		否则计算$H(x^{k})$,求$$P_{k}=-H_{k}^{-1}\bigtriangledown f\left ( x^{k} \right )$$
		
		求$\lambda _{k}$,使$$f\left ( x^{k}+\lambda _{k}p_{k} \right )=\mathop{min}_{\lambda \geq 0}f\left ( x^{k}+\lambda p_{k} \right )$$
		
		置$x^{k+1}= x^{k}+\lambda _{k}p_{k}$
		
		计算$\bigtriangledown f\left ( x^{k+1} \right )$,若$\left \| \bigtriangledown f\left ( x^{k+1} \right ) \right \|< \varepsilon $则停止计算得近似解,否则$$B_{k+1}=B_{k}+\frac{y_{k}y_{k}^{T}}{y_{k}^{T}\partial_{k}}-\frac{B_{k}\partial _{k}\partial_{k}^{T}B_{k}}{\partial _{k}^{T}B_{k}\partial _{k}}$$
		$y_{k}=\bigtriangledown f\left ( x^{k+1} \right )-\bigtriangledown f\left ( x^{k} \right )$,$\partial _{k}=x^{k+1}-x^{k}$
		
		置k=k+1,转到步骤2
	\end{algorithm}
\end{document}