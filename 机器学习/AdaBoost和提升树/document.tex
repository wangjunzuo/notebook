\documentclass{report}
\usepackage{ctex}
\usepackage{amsmath}
\usepackage[ruled]{algorithm2e}

\title{提升方法}
\author{W.J.Z}
\date{}

\begin{document}
	\maketitle
	\section{AdaBoost算法}
	对提升方法而言,有两个问题需要回答:(1):在每一轮中如何改变训练数据的权值或概率分布;(2):如何将弱分类器组合成强分类器。
	AdaBoost针对第一个问题:提高被前一轮弱分类器错误分类样本的权值,降低被正确分类样本的权值;针对第二个问题:加大分类误差率小的弱分类器的权值,减小分类误差率大的弱分类器的权值。
	
	\begin{algorithm}[h]
		\caption{AdaBoost}
		\LinesNumbered
		\KwIn{训练数据集T,弱分类算法}
		\KwOut{最终分类器}
		
		初始化训练数据的权值分布$D_{1}=\left ( w_{11},\ldots,w_{1i},\ldots,w_{1N} \right ),w_{1i}=\frac{1}{N},i=1,2,\ldots,N$
		
		对$m=1,2,\ldots,M$:
		(1):对具有权值分布的$D_{m}$的训练数据集学习,得到基本分类器$G_{m}(x)$。(2):计算基本分类器在训练数据集的分类误差率$e_{m}=\sum_{i=1}^{N}w_{mi}I\left ( G_{m}\left ( x_{i} \right ) \neq y_{i} \right )$;(3):计算$G_{m}(x)$的系数$\alpha _{m}=\frac{1}{2}\log \frac{1-e_{m}}{e_{m}}$;这里的对数是自然对数。
		(4):更新训练数据集的权值分布$$D_{m+1} = (w_{m+1,1},\ldots,w_{m+1,i},\ldots,w_{m+1,N})$$ $$w_{m+1,i}=\frac{w_{mi}}{Z_{m}}exp\left ( -\alpha _{m}y_{i}G_{m}\left ( x_{i} \right ) \right ),i=1,2,\ldots,N$$  $$Z_{m}=\sum_{i=1}^{N}w_{mi}exp\left ( -\alpha _{m}y_{i}G_{m}\left ( x_{i} \right ) \right )$$
		
		构建基本分类器的线性组合$$f\left ( x \right )=\sum_{i=1}^{M}\alpha _{m}G_{m}\left ( x \right )$$
		
		得到最终分类器:
		$$G\left ( x \right )=sign\left ( \sum_{m=1}^{M}\alpha _{m}G_{m}\left ( x \right ) \right )$$
	\end{algorithm}

	\section{提升树}
	对于二分类问题,提升树只需要将AdaBoost算法的基本分类器限制为二类分类树即可。对于回归提升树:
	\begin{algorithm}[h]
		\caption{回归问题的提升树算法}
		\LinesNumbered
		\KwIn{训练数据集}
		\KwOut{提升树$f_{M}(x)$}
		初始化$f_{0}(x)$
		对$m=1,2,\ldots,M$
		计算残差$$r_{mi}=y_{i}-f_{m-1}(x_{i}),i=1,2,\ldots,N$$
		
		拟合残差$r_{mi}$学习一个回归树,得到$T\left ( x;\Theta _{m} \right )$
		
		更新$$f_{m}\left ( x \right )=f_{m-1}\left ( x \right )+T\left ( x;\Theta _{m} \right )$$
		
		得到回归问题提升树
		$$f_{M}=\sum_{m=1}^{M}T\left ( x;\Theta _{m} \right )$$
	\end{algorithm}
\end{document}