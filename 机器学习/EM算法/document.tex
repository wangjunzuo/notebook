\documentclass{article}
\usepackage{ctex}
\usepackage{amsmath}
\usepackage[ruled]{algorithm2e}

\title{EM算法}
\author{W.J.Z}
\date{}

\begin{document}
	\maketitle
	\textbf{EM算法是含有因变量的概率模型参数的极大似然估计方法,每代迭代由两步组成:E步,求期望;M步,求极大。本文只讲述EM算法和在高斯函数上的应用,算法的推导和收敛性证明请查阅相关资料。}
	\section{EM算法}
	一般用$Y$表示观测随机变量的数据,$Z$表示隐随机变量的数据,$Y$和$Z$连在一起成为完全数据,感测数据$Y$又称为不完全数据。假设给定观测数据$Y$,其概率分布式$P(Y|\theta )$,其$\theta $是需要估计的模型参数,那不完全数据$Y$的似然函数$P(Y|\theta )$.对数似然函数$L\left ( \theta  \right )=\log P\left ( Y|\theta  \right )$.假设$Y$和$Z$的联合概率分布为$P\left ( Y,Z|\theta  \right )$,那么完全数据的对数似然函数是$\log P\left ( Y,Z|\theta  \right )$。
	
	\begin{algorithm}[h]
		\caption{EM算法}
		\LinesNumbered
		\KwIn{观测变量数据Y,隐变量数据Z,联合分布$P(Y,Z|\theta )$,条件分布$P\left ( Z|Y,\theta  \right )$}
		\KwOut{模型参数$\theta $}
		
		选择参数的初值$\theta ^{0}$,开始迭代。
		
		E步:记$\theta ^{i}$为第$i$次迭代参数$\theta $的估计值,在第$i+1$次迭代的E步,计算$$Q\left ( \theta ,\theta ^{i} \right )=\sum _{Z}\log P\left ( Y,Z|\theta  \right )P\left ( Z|Y,\theta ^{i} \right )$$ 
		$P(Z|Y,\theta ^{i})$是在给定观测数据Y和当前的参数估计$\theta ^{i}$下隐变量数据Z的条件概率分布。
		
		M步:求使$Q\left ( \theta ,\theta ^{i} \right )=\arg \mathop{max }_{\theta }Q\left ( \theta ,\theta ^{i} \right )$
		
		重复第二步和第三步,直到收敛。
	\end{algorithm}

	\section{EM算法在高斯混合模型学习中的应用}
	\noindent 高斯混合模型指具有以下形式的概率分布模型:
	$$P\left ( y|\theta  \right )=\sum_{k=1}^{K}\alpha _{k}\phi \left ( y|\theta _{k} \right )$$
	其中,$\alpha _{k}$是系数,$\alpha _{k}\geq 0$,$\sum_{k=1}^{K}\alpha _{k}=1$;$\phi \left ( \theta |\theta _{k} \right )$是高斯分布密度,$\theta _{k}=\left ( \mu _{k} ,\sigma _{k} ^{2}\right )$
	$$\phi \left ( y|\theta _{k} \right )=\frac{1}{\sqrt{2pi}\sigma _{k}}exp\left ( -\frac{(y-\mu _{k})^{2}}{2\sigma _{k}^{2}} \right )$$
	
	观测数据$y_{j},j=1,2,\ldots,N$产生过程为:首先依概率$\alpha _{k}$选择第k个高斯分布模型$\phi \left ( y|\theta _{k} \right )$;然后依第K个分模型的概率分布$\phi \left ( y|\theta _{k} \right )$生成观测数据$y_{j}$。
	
	\begin{align*}
	\gamma _{jk}=\begin{cases}
	1 , & \text{第j个观测数据来自第K个分模型} \\
	0 ,& \text{否则}
	\end{cases}
	\gamma _{jk}是0-1随机观测数据。
	\end{align*}
	完全数据的似然函数为
	\begin{align*}
	P{y,\gamma |\theta } &= \prod_{k=1}^{K}\prod_{j=1}^{N}\left [ \alpha _{k}\phi \left ( y_{j}|\theta _{k} \right ) \right ]^{\gamma _{jk}} \\
	&= \prod_{k=1}^{K}\alpha _{k}^{n_{k}}\prod_{j=1}^{N}\left [ \phi \left ( y_{j}|\theta _{k} \right ) \right ]^{\gamma _{jk}}\\
	&=   \prod_{k=1}^{K}\alpha _{k}^{n_{k}}\prod_{j=1}^{N}\left [ \phi \left ( y|\theta _{k} \right )=\frac{1}{\sqrt{2pi}\sigma _{k}}exp\left ( -\frac{(y-\mu _{k})^{2}}{2\sigma _{k}^{2}} \right )]^{\gamma _{jk}}\right ]^{\gamma _{jk}}
	\end{align*}
	其中$n_{k}=\sum_{j=1}^{N}\gamma _{jk},\sum_{k=1}^{K}n_{k}=N$。
	完全数据的对数似然函数为
	\begin{equation*}
	\log P\left ( y,\gamma|\theta   \right )=\sum_{k=1}^{K}\left \{ n_{k}\log \alpha _{k}+\sum_{j=1}^{N}\gamma _{jk}\left [ \log \left ( \frac{1}{\sqrt{2pi}} \right )-\log \sigma _{k}-\frac{1}{2\sigma _{k}^{2}}\left ( y_{j}-\mu _{k} \right )^{2} \right ] \right \}
	\end{equation*}
	确定Q函数:
	\begin{align*}
	P\left ( \gamma _{jk}|y,\theta  \right ) &= 	P\left ( \gamma _{jk}=1|y,\theta  \right )  \\
	&= \frac{P\left ( \gamma _{jk}=1,y_{j}|\theta  \right )}{\sum_{k=1}^{K}P\left ( \gamma _{jk}=1,y_{j}|\theta  \right )}\\
	&= \frac{\alpha _{k}\phi \left ( y_{j}|\theta _{k} \right )}{\sum_{k=1}^{K}\alpha _{k}\phi \left ( y_{j}|\theta _{k} \right )} 
	\end{align*}
	Q函数为:$$Q\left ( \theta ,\theta ^{i} \right )=\sum _{Z}\log P\left (  y,\gamma|\theta  \right )P\left ( \gamma _{jk}|y,\theta \right )$$
	求函数$Q\left ( \theta ,\theta ^{i} \right )$对$\theta$的极大值,分别对$\mu _{k}$,$\sigma _{k}^{2}$求偏导并令其为0即可得到解;在$\sum_{k=1}^{K}\alpha _{k}=1$条件下求偏导并令其为0得到其解。
	$$\mu _{k}=\frac{\sum_{j=1}^{N}	P\left ( \gamma _{jk}|y,\theta  \right )\gamma _{jk}y_{j}}{\sum_{j=1}^{N}	P\left ( \gamma _{jk}|y,\theta  \right )y_{jk}}$$
	$$\sigma ^{2}_{k}=\frac{\sum_{j=1}^{N}	P\left ( \gamma _{jk}|y,\theta  \right )\gamma _{jk}\left ( y_{j}-\mu _{k} \right )^{2}}{\sum_{j=1}^{N}	P\left ( \gamma _{jk}|y,\theta  \right )\gamma _{jk}}$$
	$$\alpha_{k}=\frac{\sum_{j=1}^{N}P\left ( \gamma _{jk}|y,\theta  \right )\gamma _{jk}}{N}$$
	
	\begin{algorithm}[h]
		\caption{高斯混合模型参数估计的EM算法}
		\LinesNumbered
		\KwIn{观测数据,高斯混合模型}
		\KwOut{高斯混合模型参数}
		取参数的初始值开始迭代
		
		E步:计算分模型k对观测数据$y_{j}$响应度 $$\frac{\alpha _{k}\phi \left ( y_{j}|\theta _{k} \right )}{\sum_{k=1}^{K}\alpha _{k}\phi \left ( y_{j}|\theta _{k} \right )} $$
		
		M步:$$\mu _{k}=\frac{\sum_{j=1}^{N}	P\left ( \gamma _{jk}|y,\theta  \right )\gamma _{jk}y_{j}}{\sum_{j=1}^{N}	P\left ( \gamma _{jk}|y,\theta  \right )y_{jk}}$$
		$$\sigma ^{2}_{k}=\frac{\sum_{j=1}^{N}	P\left ( \gamma _{jk}|y,\theta  \right )\gamma _{jk}\left ( y_{j}-\mu _{k} \right )^{2}}{\sum_{j=1}^{N}	P\left ( \gamma _{jk}|y,\theta  \right )\gamma _{jk}}$$
		$$\alpha_{k}=\frac{\sum_{j=1}^{N}P\left ( \gamma _{jk}|y,\theta  \right )\gamma _{jk}}{N}$$
		
		重复2和3步,直至收敛。
	\end{algorithm}
	
	
	
\end{document}